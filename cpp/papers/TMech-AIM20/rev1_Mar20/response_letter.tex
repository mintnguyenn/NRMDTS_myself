\documentclass[utf8]{article}
\usepackage{CJKutf8}
\usepackage[utf8]{inputenc}
\usepackage{comment}
\usepackage{mathrsfs}
\usepackage{amsmath}
\usepackage{blindtext}
\usepackage{amssymb}
%\usdpackage{amsthm}
\usepackage{galois}%for \comp
\usepackage{color}
\usepackage{stix}
\usepackage{bm}
\usepackage{tikz}
\usepackage{pgfplots}
\usepackage{multicol}
\usepackage{url}
%\usepackage{amscd}
\usepackage[all, pdf]{xy}

%\usepackage[paperwidth=9cm, paperheight=12cm, top=0.5cm, bottom=0.5cm, left=0.5cm, right=0.5cm]{geometry}
%\special{papersize=9cm, 12cm}
%\usepackage{geometry}
%\geometry{a4paper, scale=0.9}

\usepackage{indentfirst}
\setlength{\parindent}{2em}

\renewcommand{\thesubsection}{\thesection.\number\numexpr\value{subsection}\relax}
\setcounter{secnumdepth}{3}
%\newtheorem{definition}{定义}%[subsection]
%\newtheorem{proposition}{命题}%[subsection]
\newtheorem{theorem}{Comments}[section]
%\newtheorem{lemma}{引理}%[section]
%\newtheorem{corollary}{推论}%[section]
%\newtheorem{proof}{证明}%[section]
%\newtheorem{example}{例}%[subsection]
%\newtheorem{remark}{Remark}%[subsection]
%\newtheorem{properties}{性质}
%\newtheorem{problem}{问题}
\renewcommand{\baselinestretch}{1.5}

\title{Response Letter}
\date{March 2nd, 2020}
%\begin{centering}
%\author{Tong Yang\\ College of Control Science and Engineering\\ Zhejiang University}
%\end{centering}


\begin{document}
\thispagestyle{empty}
%\begin{CJK}{UTF8}{gkai}
\maketitle
%\newpage
%\thispagestyle{empty}
%\tableofcontents
%\newpage
%\begin{abstract}
%\end{abstract}
\noindent
Respected Editor of IEEE Transactions on Mechatronics, 

Enclosed please find the response letter for the manuscript ``Cellular Decomposition for Non-repetitive Coverage Task with Minimum Discontinuities", which we have submitted a revision for consideration by Trans. on Mechatronics. 

As attached, the reponses are described in detail following each of the reviewers' comments. 

Hopefully this effort has addressed all the reviewers' comments and made the work acceptable for your journal. 

All authors have read and approved this submission for publication. This work is original and has not been published and is not being considered for publication elsewhere in any language. 

\noindent
Sincerely, 

\noindent
Authors

\newpage
\setcounter{page}{1}

\section{Reply to Reviewer 1}
\textbf{\textit{This article investigates a cellular decomposition method for non-repetitive coverage task of manipulator. Concretely speaking, this method to optimally divide the workspace into the minimum number of cells, each traversable without discontinuities by any arbitrary coverage path within. In brief, the paper is not well written and the following suggestions need to be considered to improved the quality of the article.} }

\begin{theorem}
\textbf{As for the format of the paper, I think it is necessary to add keywords to the article so that readers can better understand the research contents. }
\end{theorem}

Thanks for reviewer's advice. We have added the keywords under the abstract, which are definitely helpful for more comprehensive understanding. In the manuscript submission system there is no option of keywords for coverage path planning algorithms, so the closest choices ``Manipulator, Planning, Manipulator motion-planning" have to be vague. Sorry for that. 

\begin{theorem}
\textbf{The proposed method can be applied to mechanical polishing and painting in industrial production, so the current application methods in the assembly line of industrial production need to be introduced to enrich the research background with more references.} 
\end{theorem}

Thanks for the reviewer's advice. We added and updated some of the descriptions in the 

\begin{theorem}
\textbf{This manuscript always focuses on inverse kinematics and joint space, but in my opinion, only the trajectory of the end-effector is considered and divided into cells. Therefore, how to generate the solution of inverse kinematics (IK) is also very important. The authors should explain the control signal (joint angle or joint velocity). }
\end{theorem}

We feel sorry about the confusion. 

In this paper, the key observation of constructing the relation between ``colour" and the configurations in the joint-space is that, for non-redundant manipulator, there is no non-singular path connecting two configurations whose EEs are at a same point. This motivates us to use different colours on a point to denote different valid configurations for this point. 
The colours are defined not only for visualization, but also for the mathematical modelling of the realistic problem: Once we have specified a colour for each point on the surface (through the proposed algorithm), we will know all inverse kinematic solutions used for the manipulator. 

Through your word, we notice the unclear description in the process of generating the topological graph, so we added another flowchart illustrating this process. 

For the assignment of the joint-angles, the relation of the joint-space configurations and the colours is mentioned in Fig. 1 caption and the third paragraph in Section III-C and used in all figures. 

As for your words ``joint velocity", we regard it as the direction of the manipulator's motion. In other words, it is about the physical pattern of the path of the end-effector on the surface (also, the configuration in the joint-space, with unique specification of the IK solutions). Althrough many other conventional coverage path planning algorithms deal with the designing of the physical path, 
what we do is a higher-level topological planning of the coverage task, the cellular decomposition. Embedded with any other conventional CPP algorithm within each cell, the physical coverage path is generated. 

\begin{theorem}
\textbf{When conducting simulations and experiments, it is generally necessary to give the parameters of the manipulator to increase the feasibility of the paper. In addition, the general UR5 manipulator is of six degrees of freedom (DOFs) or higher DOFs instead of five DOFs. Please add some description of the used manipulator. }
\end{theorem}

Thanks for the reviewer's carefulness. Yes, there is little words talking about the experiment settings, and we have added some of them at Section VI, paragraph 1. 
In order to let the Universal Robots UR5 manipulator, a 6DoF manipulator, perform like a 5DoF one, we locked its last joint. 

%As for the cells, compared with the figures that are used in discussing the algorithm process (Fig. 5-10), the purpose of the Fig. 11-16 is only to show the distribution of ``cells" appearing on object's surface. 
%The correctness of the inverse kinematic solutions are verified by the demo figure of the manipulator's configuration (i.e., (a) of Fig. 11-13, and the real-world illustration in Fig. 15, 16),  
%what we want to show in the simulated experiments is only the scene


\begin{theorem}
\textbf{Real world experiments in Section VI-C provides very little experimental process information. It is recommended to provide a link to the video of the experiment.}
\end{theorem}

Thanks for the reviewer's advice. A video illustrating the concepts and results hereby described can be found at: 

\url{https://www.youtube.com/watch?v=gyvXin60cCQ\&feature=youtu.be}

\noindent
which is also added as a footnote to the paper. 

\begin{theorem}
\textbf{The comparisons in the article is not intuitive. I suggest the author to give a comparison of the energy consumption and time consumption of different methods when conducting the experiment, if possible.}
\end{theorem}

Thanks for the reviewer's advice. Your recommendations of comparison, energy comsumption and time consumption, are classical and straightforward, but not suitable for the work embedded with the algorithms to be compared. Please see the explaination below. 

In our perspective, the two simulated experiments are designed for illustrating two advantages of using the proposed algorithm as a high-level cellular decomposition before applying the conventional CPP algorithm: Helpful in designing the relative pose between the object and the manipulator, and helpful in disregarding large amounts of joint-space configurations which cause unnecessary lift-offs of the end-effector. 

The validity of the algorithm is proved by the three equivalences given in Section IV-A, from which all possible cellular decompositions are classified into finite categories in the sense of the number of lifting-off of the end-effector. Hence the validity of the 

\begin{theorem}
\textbf{There are some minor errors in the paper that need to be corrected. For example, In line 7 of Page 1, ``Member, IEEE,," should be corrected as ``Member, IEEE,"; In line 2 Page 2, ``The remainder of this paper 1 is organised as follows" should be corrected as ``The remainder of this paper is organised as follows".}
\end{theorem}

Thanks for the reviewer's carefulness. We have amended the errors you pointed out and rechecked the whole paper hoping to make it more correct in grammar. 

We notice that the typesetting of the footnote was wrong, which caused difficulties in showing the readers the link of the suppemented video. Sorry for that. 

\begin{theorem}
\textbf{As opposed to non-repetitive path, the repetitive motion planning of manipulator is also very significant in industrial production. It is recommended that the authors can briefly compare the repetitive motion planning task with the non-repetitive path task after referring to the following papers.
\begin{enumerate}
\item A Data-Driven Cyclic-Motion Generation Scheme for Kinematic Control of Redundant Manipulators, IEEE Transactions Control System Technology, In Press.
\item On Generalized RMP Scheme for Redundant Robot Manipulators Aided with Dynamic Neural Networks and Nonconvex Bound Constraints, IEEE Transactions on Industrial Informatics, 2019.
\end{enumerate}}
\end{theorem}

Thanks for the reviewer's advice. We feel sorry about the incomplete related works. We have uploaded the section of related works which now contains the presentations about repetitive motion planning task. You can refer to the submitted paper to get the updated related works. Thanks for the reviewer's recommendation about the papers. 

xxx

We hope the new version could demonstrate the comprehensive related works. 

However, we regard the discontinuities 


\section{Reply to Reviewer 2}
\textbf{\textit{It is proposed that a mechanism to derive non-repetitive coverage path solutions with a proven minimal number of discontinuities, with the aim to avoid unnecessary, costly end effector lift-offs for manipulators. The two novel contributions of this paper include proof that the least number of path discontinuities is predicated on the surrounding environment, independent from the choice of the actual coverage path; thus has a minimum. And an efficient finite cellular decomposition method to optimally divide the workspace into the minimum number of cells, each traversable without discontinuities by any arbitrary coverage path within.}}

\begin{theorem}
\textbf{The research background needs to be more clear. It needs to extract the scientific problems to be solved, point out the problems of the existing solutions, and then draw the content of the paper to be studied.}
\end{theorem}

Thanks for the reviewer's advice. The problem of undesirable discontinuity during the coverage task really exists. The reason of such problem is the bifurcations~\cite{}~\cite{} of the joint-space and the inapplicability of the singularties in all surface task because of lack of manipulability (robustness). The Fig.1 is a visualization of the problem, and the non-optimality of greedy solutions. 
As far as the author knows, there is no already-existing solutions for generating coverage path ensuring least number of discontinuities, which is the main contribution of this paper. Also, based on our analysis of the problem (in Section III-B), it is 
\begin{theorem}
\textbf{Combined with actual application scenarios, rather than directly abstracting into a mathematical model, it needs to be realistic.
}
\end{theorem}

Theoretically, the problem of undesirable discontinuities during coverage task exists in all applications using manipulators with revolute joints, and the proposed algorithm is applicable in any dimension as long as the manipulator is non-redundant and the 

\begin{theorem}
\textbf{Language should be easy to understand, not too much advanced vocabulary.
}
\end{theorem}


\begin{theorem}
\textbf{The experiment should be further improved to prove the algorithm, not a simple pot surface polishing.
}
\end{theorem}

Thanks for the reviewer's advice. We 

\section{Reply to Reviewer 3}
\textbf{\textit{
The paper presents a novel method of cellular decomposition by which a manipulator can perform a task in the workspace with minimum number discontinuities or lift offs of the end effector. While dividing the workspace to number of cells for the task, they also proved that the number of discontinuities or lift offs are not affected by the type of coverage path adopted. They have also shown the application of it through simulation and a real world experiment with a manipulator.
}}

Thanks for the reviewer's approvement of the proposed algorithm. Through your comments, we noticed that there are some inadequacies in describing the ``solutions" and ``all optimal solutions". We have added them.  

\begin{theorem}
\textbf{
For different topological graphs different configurations are chosen. I would like to know how these configurations are chosen? Is it that there are a number of solutions of configurations to cover the cells and any one is chosen or there is really a unique solution of a configuration.
}
\end{theorem}



\begin{theorem}
\textbf{I would also appreciate a flow chart kind of thing starting from assignment of the task to the execution of the task by the robot to be there. Right now, it is not so clear. That would make it complete.}
\end{theorem}

\begin{theorem}
\textbf{I would also like to know if the entire thing can be automated or not? End to end. A brief description about this will be helpful.}
\end{theorem}


\begin{theorem}
\textbf{There are many grammatical errors in the paper. I have highlighted a few in red as you will find in the attached paper. Please address them also.}
\end{theorem}

\begin{theorem}
\textbf{I have also asked a few more questions in the attached file and highlighted some of the grammatical errors. I would suggest you to please go through the manuscript thoroughly.}
\end{theorem}

Thanks for the reviewer's carefulness. We have amended the errors you pointed out and rechecked the whole paper hoping to make it more correct in grammar. 

In our algorithm, the solution of the cellular decomposition of the given surface is a colouring scheme of the corresponding topological graph. Obviously, there are many different topological cellular decompositions using the 

As for the example in Section VI-B, since 



\newpage
%\end{CJK}
\end{document}


