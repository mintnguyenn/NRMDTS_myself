\documentclass[11pt]{article}
\usepackage[framemethod=TikZ]{mdframed}
\usepackage{setspace}
\onehalfspacing
\usepackage{fullpage}
\usepackage{url}

\usepackage{graphicx}
\usepackage{amsmath,amssymb,enumerate,caption} %thmtools,
\usepackage{booktabs}
\usepackage{verbatim}
\usepackage{color,soul}
\usepackage{hyperref}

\newenvironment{reviewer}
{\begin{mdframed}[roundcorner = 10pt,fontcolor=blue!70!black]\itshape}
{\end{mdframed}}

\newcommand{\bigspace}{\vspace{0.1cm}}

\author{Jaime Valls Miro \and Tong Yang}

\begin{document}

\setlength{\parindent}{0pt}
\hspace*{0.5\linewidth}
\begin{minipage}{0.7\linewidth}
Dr. Tamim Asfour, Professor \par 
Editor-in-Chief \par 
IEEE Robotics \& Automation Letters \par 
%Fellow of Academy of Engineering, Singapore \par 
%Fellow of IEEE \par
%Fellow of ASME \par 
Institute for Anthropomatics and Robotics \par 
Karlsruhe Institute of Technology \par
%School of Mechanical and Aerospace Engineering \par
%Nanyang Technological University \par
%2145 Sheridan Road, B222\par
%Evanston, IL 60208\par 
\bigskip
January 13, 2022 % Change to re-submission date
\end{minipage}

\vspace{2cm}


Dear Prof Tamim Asfour, \par \bigskip

We would like to thank you, the editors, and the reviewers for the helpful feedback on the revision of our manuscript. As instructed, we are now submitting a revised version of our paper (IEEE Robotics \& Automation Letters: 21-0690)\footnote{``Optimal Task-space Tracking with Minimum Manipulator Reconfiguration'' by Tong Yang (255613), Jaime Valls Miro (106478), Yue Wang (156231) and Rong Xiong (113216)}. 
The paper has been modified according to your feedback and the Reviewers' comments. 
Please find our detailed responses in the following pages. 
%In addition, we have also highlighted the changes in the revised version of our manuscript in \textcolor{blue}{blue}.

\par \bigskip All authors have read and approved this submission for publication. This work is original and has not been published, or is being considered for publication, elsewhere in any language. 

\par \bigskip Once again, thank you for your time and kind consideration. 

\bigskip
Sincerely,
\par \bigskip
Tong Yang \par 
Jaime Valls Miro \par
Yue Wang\\
Rong Xiong\footnote{Corresponding author (\texttt{rxiong@zju.edu.cn})} \\

\clearpage

\section*{Response to the Editors}
%We would like to thank the editors for the helpful suggestions on the revised manuscript. 
Please find our responses to the point raised by the editors below. 
In the following sections, we answer the individual reviewers' questions in detail (\textcolor{red}{R1} and \textcolor{red}{R2}).

\begin{enumerate}[I.]
\begin{comment}
 \item $ $
  \begin{reviewer}
  On the basis of the reviewers' ratings and comments, we decided that your paper cannot be published in the Letters in the present form. However, you are encouraged to rewrite and resubmit a revised version of your work addressing all editorial concerns. 
  \end{reviewer}
\end{comment}
\item $ $
  \begin{reviewer}
%The authors have proposed a method for finding a joint space trajectory for following a task-space trajectory. 
The reviewers thought the work was good and interesting, but there were some gaps in the experiments that should be resolved before the final version. 
  \end{reviewer}
  \noindent
  We added more experiments and included further metrics to the analysis of the results as suggested by both reviewers. 
%We details of the proposed algorithms and clarity of presentation, .
%The detailed response to each of the reviewer's comments are listed next. 
%Thank you for the feedback and consideration.
\end{enumerate}


 \clearpage

\section*{Response to Reviewer 1}
We thank Reviewer 1 for the careful examination of the manuscript and the helpful feedback and suggestions. Please see our detailed responses below.

\begin{enumerate}[I.]

 \item $ $
  \begin{reviewer}
    One major issue I found with this paper is that the proposed solution does not account for the path taken by the RRT during the switches. This could lead to sub-par performance, affecting the cycle time and the total path length which were the initial motivations for the paper as mentioned in the introduction. I think incorporating the RRT-path in the algorithm is non trivial and could in itself be a research paper. 
\begin{comment}
For this submission, it would be good to run some experiments in highly constrained environments to see if the total path length (joint-space) is different across many optimal solutions produced by the proposed method.
Specifically reporting the path length during the task-space tracking and rrt-path for switching separately will be very insightful. 
Additionally reporting time taken on the real robot will help show the improvement with the proposed method. 
\end{comment}
  \end{reviewer}


The reviewer is correct in his observation that the actual RRT planner adopted during the switches is not specifically accounted for in the algorithm. 
This would indeed call for a new research paper in itself, and we thank the Reviewer for the pointer that we leave for future work.
Yet this being the case for any of the reconfigurations motions adopted by any of the planners, optimal or sampled, it is felt the comparison and advantage of the proposed scheme with respect to minimal reconfigurations remain validated. 


% TONG: Reviewer is not saying THIS I believe??? no need to mention. PLS read question carefully in case I misunderstoodm but I dont think so. If not asked, we dont need to answer.
%The reviewer is right that the optimal solutions are non-unique. 
%For the testing cases where multiple optimal solutions exist, we reported two of them, so that the comparison between optimal solutions becomes straightforward. 
%The proposed algorithm can find all optimal solutions, and the users may choose the best one among them, where the ``optimality" may be under a composite consideration of the number of reconfigurations, size of bounding boxes, task-space/joint-space travelling distance, etc. 

\begin{comment}
\begin{color}{red}
I think both your words and mine are required. When seeing ``\textit{the proposed solution does not account for the path taken by the RRT during the switches}", I thought the same as you, i.e., reviewer 1 asked about whether there is a non-min-reconfiguration solution with a faster execution time, the same as reviewer 2.1. But then he said ``\textit{For this submission, it would be good to ... to see if the total path length (joint-space) is different across many optimal solutions ...}", where he was indicating a comparison between optimal solutions. This is why I reported two instances of "ours" in case 2 and case 4 in Table 2. Though not a critical issue from the reviewer, I think we should mention ``we have done this". 
\end{color}
\end{comment}

 \item $ $
  \begin{reviewer}
   For this submission, it would be good to run some experiments in highly constrained environments to see if the total path length (joint-space) is different across many optimal solutions produced by the proposed method.
Specifically reporting the path length during the task-space tracking and rrt-path for switching separately will be very insightful. 
  \end{reviewer}

(This query is also somewhat related to R2.I and R2.II, albeit not specifically for the optimal solutions as hinted by Reviewer I. The Reviewer is referred to the answer thereby provided also).

%We thank the Reviewer for the suggsestion. we have done this and agree makes the argument possibly compelling.
We have added a new simulated comparison (case 4) with more clutter in the environment. We have added and reported on new metrics as suggested by the Reviewer, where the length traversed for all testings cases is considered, including separately reporting on the final relative lengths of the  EE with respect to the desired task-space tracking paths, thus accounting for EE deviations during reconfigurations (with the RRT implementation adopted). This is extended not only to the optimal solutions, but also to all the comparative cases.
The metric is reported both in task- and joint-space lengths for completeness. It is seen how when it comes to the number of reconfigurations, the proposed scheme solves for the optimal solution, yet total lengths do not necessarily remain shorter in those cases, as was not regarded in the optimisation factor. They often do though. 
This is now even more apparent when looking at the numbers reported in the table and the added discussions to Section V.B. 

We are showing in a new table (II) two sampling-based resulting motions for each comparative case (random 1-8). Moreover, as pointed by the Reviewer, there may be more than one optimal solution when it comes to the number of reconfigurations, and have added that as well to the table. Not all solutions are shown for brevity: two optimal solutions are shown for cases 2 and 4, where more than one solution exists (case 1 and case 3 have a singular optimal solution). The number of optimal solutions has also been incorporated into the table for added clarity. 
All these results are compiled in the new,  much expanded Table II (Table I in our initial submission). 

It is noteworthy (stated in the manuscript) that we adopted a constant joint-space velocity model, so that the comparison between execution time is equivalent to the comparison between joint-space travelling distance. 

\item $ $ 
  \begin{reviewer}
  Additionally reporting time taken on the real robot will help show the improvement with the proposed method. 
  \end{reviewer}

The execution times of the real-world robot had been shown in the supplemented video in our initial submission, but have now also been added to the revised paper as advised (Section V.B). 
Moreover, we have also added the metric to all the simulation cases in Table II.


  \item $ $
  \begin{reviewer}
   The paper mentions a supplementary video which was not submitted. 
  \end{reviewer}

\noindent

There was a mistake in the embedded http reference in the latex code, our apologies for the oversight. 
While the video http address stated was correct and could be copied and pasted to visualise the video, the embedded link was not. This has now been fixed.  
%A comprehensive video illustrating the concepts and experiments described in the paper had already been supplied in the original manuscript. 
%It was detailed in footnote 2, at the end of Section I. 
%It appears the reviewer may have missed the reference. 
The video in our initial submission can still be found at: \\
\url{https://github.com/ZJUTongYang/min_reconfig_taskspace_tracking_video}

%The video URLs have also been embedded in the pdf so that it is .
The video has been further edited to add enhancements as suggested by the reviewers with additional experimental examples and modifications to the actual visualisations 
to add clarity in some of the simulations (See R1.VI reviewer comment below). The improved version, directly hyperreferenced in the manuscript, can be found at: \\
\url{https://youtu.be/HHqGBk9_3x8}
  
\newpage

  \item $ $
  \begin{reviewer}
  Having a table for different variables used will be helpful. 
  \end{reviewer}

\noindent
%Thank you for the advice. 
We have incorporated Table I listing the notation for the most repeatedly-used variables. 
  
  \item $ $
  \begin{reviewer}
    In Fig.5 and 6, the grey trajectory (rrt path) could be shown with a dotted line to improve visual clarity. 
  \end{reviewer}
\noindent
%Thank you for the advice. 
We have shown the reconfiguration motions as dotted curves, and also further reduced the width of the curves during the reconfiguration phases. 
We thank the reviewer as it indeed increases the clarity of the visualisation. 

  
\end{enumerate}

\clearpage

\section*{Response to Reviewer 2}
We thank Reviewer 2 for the careful examination of the manuscript and the helpful feedback and suggestions. Please see our detailed responses below.

\begin{enumerate}[I.]

  \item $ $
  \begin{reviewer}
The authors develop an algorithm for globally minimizing the number of reconfigurations during pose trajectory following as a way of improving behavior during 
these unavoidable maneuvers. The authors test on a few cases showing how the optimization process does indeed reduce the number of configurations as 
opposed to a randomly selection baseline.

  The main question I have regarding the approach is whether solving for the minimum number of reconfigurations actually guarantees improvement over the performance metrics 
cited, i.e., reducing workspace volume requirements, reducing motion times and energy cost. 
It is true that their approach reduces the number of reconfigurations, but the results also show that the random baseline can produce more reconfigurations with less workspace volume (see Case 2). I also found the execution times for maneuvers of the random baseline to be competitive at times with the optimized solution. 
   \end{reviewer}

\noindent
%Thank you for the advice. 
The Reviewer's appreciation is correct in that optimality of the proposed work is prescribed to the minimum number of EE deviations from the pre-defined path, and that indeed does not guarantee optimality with respect to other parameters. The goal seeks to minimise the need to adopt reconfigurations for specific tasks where that discontinuity is notably undesirable, e.g. welding or painting. When extending to other metrics, optimal minimum reconfiguration paths might not translate into shortest or fastest paths (albeit they often do). 
The fact that the result of case 2 (in our initial submission manuscript and video) showed a solution that exhibits a smaller bounding box and a shorter joint-space travelling distance than the minimum reconfiguration optimal path is testament to the objective of the authors to show the results in full light, whilst the key objective remains to reveal the optimality of the solutions found with respect to arm reconfigurations. 

Having said that, we concur with Reviewer 2 - which also follows from a similar comment from Reviewer 1 - that a broader set of metrics and test cases will aid in providing a fuller 
picture of the capabilities of the proposed planner in terms of motion planning in general. %In fact, they reflect the ``min-discontinuity" paths. 
In that regard, and following the advice (also from Reviewer 1), Table II incorporates a more extensive set of metrics that make this manifest, and the extended discussion in Section V.B further reveals the scenarios raised by the Reviewer. All the joint- and task-space path lengths for all motions are reported, as well as execution times, thus extending the analysis of the test cases beyond the pure number of reconfigurations that is core to the manuscript. 
Besides, the distribution of the number of reconfigurations in all possible greedy solutions is visualised as a bin histogram chart in Table II. It can be seen that for example, in case 2, a solution with 11 pose reconfigurations has the highest probability to be constructed by sampling-based strategies; the proposed algorithm is able to solve for the solutions with 1, the minimum.

\begin{comment}
\textcolor{red}{TO BE REMOVED: (T) I'm not sure whether the histogram should be referred as ``empirical", both here and in the paper: For each test case, I calculated the number of reconfigurations of all greedy solutions and build the histogram, instead of randomly choosing 100 cases among them. (J) It is empirical in that that it is exactly how youu have constructed the distribution, and we don't provide the real underlying distributon, but the histogram of cases, which is by definition the empirical distribution. I'm being a bit pedantic here to be as accurate as possible, but bottom line is we don't have a the continuous, real distribution of the variable "lift-offs" - that we could draw from the histogram, but we don't; what we have is the empirical distribution, which is a histogram of their discrete values.
(T) I think the distribution is precise and we should remove ``empirical". I counted the lift-offs for all 6084964 greedy solutions, and drew the histrogram. The distribution of "lift-offs" number is discrete, 2 or 3 or ..., impossible to have a motion with 2.5 lift-offs. So the distributions are exactly the ones shown in the paper. (J) Agree}
\end{comment}  

%We have detailedly reported the data in Table II in the revised paper (random 3 testing). 
%In fact, they reflect the ``min-discontinuity" paths. 
%In practical usage, the users would freely adopt any trade-off between criteria: the time to completion, the energy consumption, the number of discontinuities, etc. 
%One of the most illustrative applications is the welding task, where the discontinuities significantly influence the quality of the task, and minimising the number of discontinuities then outweighs any other metrics, i.e., the value of min-reconfiguration solutions stand out. 

%We have detailedly reported the data in Table II in the revised paper (random 3 testing). 
%However, it is a small-probability event that the sampling-based planner can generate a solution with a near-optimal number of reconfigurations. 

  \item $ $
 \begin{reviewer}
To address the above issue, I would recommend evaluating across a much larger number of cases, where the 3D convex hull of motion, execution time, etc., are reported and compared between the proposed solution and the baseline random one. 
   \end{reviewer}

We have extended the analysis with a more complex case (4), where more clutter has been added to the environment. 
\begin{comment}
\textcolor{blue}{This is of greater relevance to the difficulty in the solution when additional obstacles are present that the actual nature of individual obstacles in relation to their enveloping hull. }
\end{comment}
New metrics have been added, including a bounding box for the resulting motion, execution times, and lengths relative to the desired path to track (please refer to response R1.II).
All test cases have been evaluated exhaustively, with the full results gathered in Table II, and extended examples included in the adjoining video. %, with all the possible cases and solutions with respect to lift-offs captured in the new empirical histogram in Table II. 
Moreover, we have incorporated two solutions generated by the random sampling-based strategies for each testing case, also visualised in the video for better context and clarity. 
We have also added values for 2 optimal solutions each for cases 2 and 4, further drawing attention via the example studied to the fact that more than one solution may exist.  

%Please refer also to the response provided to R1.II comment, also related to this query.

  \item $ $
  \begin{reviewer}
	There are spelling errors throughout that need to be fixed. 
   \end{reviewer}

\noindent
We have fixed all the grammatical omissions we could identify, and believe to be free of any misspellings. 
%We thank the Reviewer for the time and detailed consideration. 


\end{enumerate}


\end{document}
