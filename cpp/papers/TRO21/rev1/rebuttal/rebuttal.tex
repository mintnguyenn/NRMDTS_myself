\documentclass[11pt]{article}
%\usepackage{tcolorbox}
\usepackage[framemethod=TikZ]{mdframed}
%\tcbuselibrary{skins}
\usepackage{setspace}
\onehalfspacing
%\usepackage{icomma}
%\usepackage[hidelinks]{hyperref}
\usepackage{fullpage}
\usepackage{url}

\usepackage{graphicx}
\usepackage{amsmath,amssymb,enumerate,caption} %thmtools,
\usepackage{booktabs}
%\usepackage[caption=false,font=footnotesize]{subfig}
%\usepackage{flushend} % to align the last page
\usepackage{verbatim}
\usepackage{color,soul}
%\usepackage{dirtytalk}
%\usepackage{hyperref}


\newenvironment{reviewer}
{\begin{mdframed}[roundcorner = 10pt,fontcolor=blue!70!black]\itshape}
{\end{mdframed}}

%\newenvironment{reviewer}{\begin{tcolorbox}[enhanced,frame
% hidden,width=\textwidth,colback={red!60!white}]}{\end{tcolorbox}}
%\newenvironment{reviewer}{\begin{tcolorbox}[enhanced,frame
% hidden,width=\textwidth,colback={green!60!white}]}{\end{tcolorbox}}
\newcommand{\bigspace}{\vspace{0.1cm}}

\author{Jaime Valls Miro \and Tong Yang}

\begin{document}
%\noindent Dear Prof.~Park, Associate Editor and Reviewers,\\[0.3cm]
%On behalf of my coauthors and myself, I would like to thank you for your time, the in-depth reviews and helpful
%advice. Please find our detailed responses below. As instructed, we only made
%minor changes to the paper (e.g., typos, missing references).
%\\[0.3cm]Kind Regards,\\


\setlength{\parindent}{0pt}
\hspace*{0.62\linewidth}
\begin{minipage}{0.7\linewidth}
Dr. Kevin Lynch, Professor \par 
Editor-in-Chief \par 
IEEE Transactions on Robotics \par 
%Fellow of Academy of Engineering, Singapore \par 
%Fellow of IEEE \par
%Fellow of ASME \par 
Center for Robotics and Biosystems\par 
Northwestern University\par
%School of Mechanical and Aerospace Engineering \par
%Nanyang Technological University \par
2145 Sheridan Road, B222\par
Evanston, IL 60208\par \bigskip
April 12, 2022 % Change to re-submission date
\end{minipage}

\vspace{2cm}


Dear Prof Lynch, \par \bigskip

We would like to thank you, the editors, and the reviewers for the helpful feedback on the revision of our manuscript. As instructed, we are now submitting a revised version of our paper (IEEE Transactions on Robotics: 21-0448)\footnote{``An Improved Maximal Continuity Graph Solver for Non-repetitive Manipulator Coverage Path Planning'' by Tong Yang (255613), Jaime Valls Miro (106478), Yue Wang (156231) and Rong Xiong (113216)}. The paper has been modified according to your feedback and the Reviewers' comments. Please find our detailed responses in the following pages. 
In addition, we have also highlighted in \textcolor{blue}{blue} the primary passages that have been altered in the revised version of our manuscript to address the feedback received.

\par \bigskip All authors have read and approved this submission for publication. This work is original and has not been published, or is being considered for publication, elsewhere in any language. 

\par \bigskip Once again, thank you for your time and kind consideration. 

\bigskip
Sincerely,
\par \bigskip
Tong Yang \par 
Jaime Valls Miro \par
Yue Wang\footnote{Corresponding author (\texttt{wangyue@iipc.zju.edu.cn})}\\
Rong Xiong \\

\clearpage

\section*{Response to the Editors}
%We would like to thank the editors for the helpful suggestions on the revised manuscript. 
Please find below %our responses to the points raised by the editors, followed by 
a summary of changes in our revised manuscript. In the following sections, we answer the individual reviewers' questions in detail (\textcolor{red}{R1}, \textcolor{red}{R3} and \textcolor{red}{R3}).

\begin{enumerate}[I.]
 \item $ $
  \begin{reviewer}
%On the basis of the reviewers' ratings and comments, we regret to inform you that your paper in the present form cannot be published in the Transactions. However, you are encouraged to submit a revised version of your work addressing the reviewers concerns. 
%The revised paper will be handled as a new submission that will be reviewed accordingly while keeping track of the previous review material. 
  The revised manuscript should be formatted so that any changes can be easily identified by the reviewers, by using, e.g., coloured or bold text to indicate revised passages.
  \end{reviewer}

\noindent
Modifications are shown in \textcolor{blue}{blue}. When whole sections/paragraphs may have been mainly re-arranged to attend to a clearer organisation as recommended by the reviewers, we have indicated so below, but chosen not to colour these full blocks in their new location to increase the readability of the revised manuscript.

\item $ $
  \begin{reviewer}
% This paper is about non-repetitive coverage planning with maximal continuity performed by a non-redundant manipulator. 
%The approach is based on graph colouring, with a reduction in complexity by accounting for certain topological invariants. 
%Though the reviews note certain merit of the paper, particularly with regard to the overall approach, they also call attention to serious issues with the manuscript. 
%Thus, a thorough revision is needed for the paper to have a chance of being a suitable for publication in the Transactions on Robotics. 
Most importantly, the reviews note that the organization and presentation of the paper can often be difficult to follow, to an extent that the technical content is challenging to understand to evaluate. 
Thus, the authors are encouraged to: 

\begin{enumerate}
\item Ensure that the manuscript is self-contained, particularly in the sense of being intelligible for readers that have not read the authors' prior work in reference 23. 
This will require, among other things, a precise statement of the problem to be solved in terms of its inputs and outputs, and clear and precise definitions of all of the relevant terminology within the main text of this paper. 

\item Deeply revise the technical exposition, particularly with an eye toward ensuring that the definitions, lemmas, theorems, and other elements are precise, mathematical statements. 

\item Restructure the introduction, to ensure that the motivation for the work is persuasively presented, that contributions of the paper are directly stated, and that the structure of the introduction is clearly discernible. 
\item Expand the evaluation to include more complex objects that more clearly demonstrate the effectiveness of the approach. 
\item Update the references to include more recent results on coverage path planning, and to refer to the latest versions of work that has appeared in journals. 
\end{enumerate}

%In addition, the reviews provide a number of specific suggestions for additional clarification and correction that the authors should address. 

  \end{reviewer}
  \noindent
 We carefully addressed the issues raised by the reviewers regarding details of the proposed algorithms, clarity of presentation, technical statements, and further experimental results. These changes are listed in corresponding order in the summary of changes next, as well as in our detailed response to the reviewers that follows. %Thank you for the feedback and consideration.
\end{enumerate}

\subsection*{Summary of Changes:}

\begin{enumerate}
\item[$(a)$] Based on the editorial and reviewers' comments for self-containment, we have moved and expanded the content previously included as an appendix - where key background information was supplied - to the main body of the manuscript (Section IV), and added necessary examples to make the manuscript more intelligible as a whole.
Besides, we have added a glossary of key terms raised in the main body of the manuscript as an appendix. 
(\textcolor{red}{R1}, \textcolor{red}{R2}, \textcolor{red}{R3})
\item[$(b)$] We fixed, improved, and where possible simplified notation and language throughout the paper. All definitions are provided before they are first used. Illustrative examples are no longer listed as lemmas to avoid possible misunderstandings. 
(\textcolor{red}{R2}, \textcolor{red}{R3})
\item[$(c)$] We have re-arranged the long original introduction section, structuring it into a number of subsections to ensure the final manuscript is more self-contained, and the key contributions easier to appreciate from the onset. The new version includes stronger motivations for the NCPP problem (Section I), related works (Section II), problem statement (Section III), and a more detailed recounting of the existing state-of-the-art full solver for the NCPP task that the proposed scheme improves on (Section IV).  %The motivation for solving the NCPP task is presented in detail in Section I. 
(\textcolor{red}{R2}, \textcolor{red}{R3})

A separate list of contributions is provided in its own subsection (Section IA) at the end of the introduction. 
(\textcolor{red}{R2})
\item[$(d)$]  We have added two new testing cases (represented by arbitrary meshes instead of structured meshes) as further examinations. Besides, we have added a video to the submission where more extensive examples of the experiments are illustrated, including animations of the resulting manipulator NCPP paths on the objects shown in the manuscript. 
We have compared the computational time of the proposed improved solver against the existing algorithm it supersedes.
(\textcolor{red}{R1}, \textcolor{red}{R3})
\item[$(e)$] We have mentioned more recent research works, and switched conferences to their journal version where we could identify a journal version of the work. (\textcolor{red}{R3})

\item[$ $] Additionally: 


\item[$(f)$] An open source implementation is provided in this revised manuscript as a new contribution. 
\end{enumerate}
 \clearpage

\section*{Response to Reviewer 1}
We thank the Reviewer for his/her positive review. 
Please see our detailed responses below.

\begin{enumerate}[I.]

 \item $ $
  \begin{reviewer}
    A paragraph on the original NCPP would be a nice addition to this section.  
  \end{reviewer}

\noindent
%Thanks for the advice. 
Thanks, this was also suggested by the other reviewers. We have now added a clear NCPP problem statement definition in Section III, and a full detailed summary of the existing enumerative solver for the NCPP task in its own Section IV, consolidating the NCPP background information originally supplied as an appendix in the initial submission as part of the main body of the manuscript in the revised version.
  
  \item $ $
  \begin{reviewer}
   The experimental results are based on the simulation on a hat-like concave hemisphere and a saddle surface. It looks like though from the perspective of segmentation the manipulator motion is not very intuitive. Maybe a more complex object would have been more beneficial to consider in this case, since the hat-like shape for example intuitively could probably be painted with only one colour with circular motion. But if it is not the case, it would be great to elaborate on why those shapes are good test sample examples. 
  \end{reviewer}

\noindent
%Thank you. 
We have incorporated two additional testing cases on unstructured meshes in the experimental section, and extensively illustrated all the results, including animations of the resulting paths, in an additional video added to the submission. 
In the supplementary video we also depict how a naive spiral motion that may intuitively seem simplistic in task space on the hat-like object. 
It nevertheless results in motions with abundant lift-offs due to singularities, self-collision, and unreachable/low manipulability along the path.
Moreover, as required by Reviewer \#3, we have also added the comparison of computational time between the proposed improved solver and the existing algorithm to show the algorithmic complexity improvement. 

%As for the hat-like example, given the collision avoidance constraint and the manipulability constraint, it is non-trivial to solve. \textcolor{red}{The naive circular motion discarding will have X end-effector lift-offs, versus the current optimal solution with Y lift-offs. }
%Please refer to the supplemented video for an illustration. 
  
\item $ $
\begin{reviewer}
Throughout this paper, the algorithm is referred to as NCPP, but at the beginning it was defined as the original version from paper [1]. Maybe it could be better to use different terms for the proposed approach in this paper? 
\end{reviewer}

\noindent 
Thanks for the advice, it indeed adds clarity to separately nominate the proposed solver. We have named it ``IMCGS" (Improved Maximal Continuity Graph Solver), and referred to it throughout the manuscript. Please note the NCPP is the generic problem, which remains the same; it is a solver aimed at achieving maximal continuity in the resulting coverage path that we strive to solve for, and do so more efficiently with a novel take on the graph modelling employed than it is possible with other contemporary schemes.
  
\item $ $
\begin{reviewer}
On page 10, when mentioning naïve enumeration proposed in the literature, please put a reference to that work. 
\end{reviewer}

\noindent
%Thanks for the advice. 
The naive enumeration approach is our previous work, citation [23] in our initial submission. We have added the citation in the revised manuscript. 

\item $ $
\begin{reviewer}
On page 10, please elaborate in short what do you mean by minor abuse of notion in this context. 
\end{reviewer}

\noindent
We felt that the way we showed the multiplication terms might be a little lengthy. Now that it is unnecessary, we have removed it for clarity. 

  \item $ $
  \begin{reviewer} 
    In the appendix, Lemma 22 is without proof. It will be useful to have reference to the proof if it is available in literature. 
  \end{reviewer}

\noindent
Thanks for the advice. 
Lemma 22 was proven in [23] (in our initial submission). 
We have cited the paper for proofing in Lemma 7 (a new number in this submission, as it has been moved into the main body of the text as part of the expanded background work in Section IV, as suggested). Further explanations of the original solver are included - for completeness, as adviced -  including illustrations in Fig.3 - new - and  Fig.4, together with a concrete example showing the enumeration process (Eqn.(5) and Eqn.(6)). 
  
  \item $ $
  \begin{reviewer}
    Grammar Issues: 
\begin{enumerate}
\item  Page 2: posissible solutions $->$ possible  solutions
\item  Page 8: Fig 13 is referencing to Fig 14 that comes after, possibly change the order.
\item  Page 10: results for the others $->$ results for the other
\item  Page 11 : Earlier work has proven that [23] the number $->$ Earlier work [23] has proven that the number
\end{enumerate}
  \end{reviewer}

\noindent
%Thanks for the advice. 
%We have attended to these, thanks. We have also removed the reference in the caption of Fig.13 (in the initial submission) to avoid cyclic reference, and 
Thanks for pointing these out, we have fixed all these grammatical omissions. 
  
\end{enumerate}

\clearpage

\section*{Response to Reviewer 2}
We thank the Reviewer for the careful examination of our response and the revised paper, and the helpful feedback and suggestions. Please see our detailed responses below.

\begin{enumerate}[I.]
 \item $ $
  \begin{reviewer}
    The manuscript is not self-contained in its current form. The paper cannot be understood fully without reading reference [23] which is authors previous work on the problem. 
The authors don't even include an explicit problem formulation in the paper. 
Important building blocks for the problem have been pushed to the appendix and should be included within the main text. 
    \end{reviewer}

\noindent
%Thanks for the advice. 
The manuscript has been substantially restructured as advised, and made fully self-contained. 
We have explicitly incorporated a problem statement (Section 3) and a summary of the existing solution (Section 4) in the main text. 
The introduction section has also been reorganised to further motivate the work and make the contribution clear to the reader.
  
  \item $ $
  \begin{reviewer}
It is hard to follow the flow of ideas in the current draft. 
The writing can be considerably improved to make it concise and highlight the creation of intersection-free regions for cell decomposition and painting. 
Authors should consider moving most of the proofs to the appendix to avoid disrupting the flow. 
   \end{reviewer}
  
\noindent
Thanks for the advice.
We have restructured the paper for increased clarity. We hope to have highlighted more effectively this way the main propositions in Section V and VI about the key concept of intersection-free regions and strips, also in the supplementary video. And make these concepts flow better now. 

To further ameliorate flow disruptions some explanations (formally written as proof in our initial submission) that were specific to examples given have been moved to figure legends instead ( Fig.4 to 6), and we have removed Fig.7, Fig.8, and Fig.11 (in our initial submission), replaced by a single Fig.6 (in our revised manuscript). 
A single short reference to Figs.7 to 9 (in our revised manuscript) has been added (last paragraph of Section V) to succinctly show the shortcomings of the existing graph enumeration solution, where before these were more spread out, thus avoiding disrupting the flow and making the challenge and contribution stand out clearer. 

We have retained the proofs in the main body of the manuscript as we have made them as succinct as possible and feel it would have been tedious to refer back to an appendix to revise them. We hope the reviewer agrees with this call in light of the other changes imposed to add flow and clarity. We feel by remaining closer to where they are introduced, readability is enhanced.

%Based on the reviewer's feedback, we have reconsider some sections in pursuit of further clarity. The examples  in Fig.4 to 8, and Fig.11 in the initial submission are critical counterexamples that reveal the difficulty of solving NCPP task, the explanations (formally written as proof in our initial submission) were specialised to the examples, and now we realised that some of them did not directly contribute to the algorithm. 
%Hence, as suggested by your comments and Reviewer \#3's comments, to avoid unnecessary complexity, we move the explanation of Fig.4-6 to their figure legends, and removed Fig.7, Fig.8, and Fig.11 (in our initial submission), replaced by Fig.6 (in our revised manuscript). 


  \item $ $
  \begin{reviewer}
    The paper also has many grammatical errors and long unnecessarily complicated sentences, that make it hard to read and confuses a reader. For instance: 

- Corollary 7: No algorithm that can solve a graph with explicit intersection-decreasing, such as transforming an intersection-n graph to an intersection-(n-1) partly-solved graph. 

- It can be thus concluded that any graph solvers (including all existing solvers, and the one to be proposed in this work) will face an enumerative scheme, where there is no guarantee of an explicitly decreasing number of intersections as a graph gets gradually solved. 

- Lemma 22: Different parts of a cell may be painted with different colours provided that the design of the cell cutting paths satisfy that: 

\qquad(1) It is sufficient to consider cutting paths that start and end at the edge endpoints.

\qquad(2) It is unnecessary to consider cutting paths that go across edges. 

\qquad(3) It is unnecessary to consider intersecting cutting paths. 
   \end{reviewer}

\noindent
We have fixed all these grammatical omissions. Also others picked up by Reviewer \#1. We thank the reviewer for the time and detailed consideration.   
  
  \item $ $
  \begin{reviewer}
    Can the authors comment on the trade-offs to achieve this computationally efficient NCPP solution with minimal lift-offs. It will be interesting to see how the approach developed in this paper affects the path length as compared to other approaches to NCPP or that of [23]. An ideal solution should not affect the path length since the coverage is non-repetitive. 
   \end{reviewer}

\noindent
The reviewer is correct that the improved solver does not affect the quality of the resulting path. 
There is no trade-off. 
We have written this explicitly in the second term of the contribution list in the introduction, Section IA. 

  \item $ $
  \begin{reviewer}
    The introduction does not motivate the problem sufficiently. The text starting from “Given the object’s surface, the manipulator, the surrounding ...” to “... critical such as painting [4], deburring [5], welding [6], scanning [7], etc” justifies why lifting is undesirable. However, it does not make it clear why non-repetitive coverage is useful in the given context. 
   \end{reviewer}

\noindent
%Thanks for the advice. 
We have added some further motivations to the work in the first paragraph of Section I. 
Finding a non-repetitive coverage path solution is critical when the object surface to be manipulated is most sensitive to over-coverage, e.g. material removal via jet blasting.  The NCPP task becomes non-trivial when a joint-space continuous full-coverage motion is not readily achievable when using a manipulator for such handling tasks. 
%We have added several applications in this regard in the introduction. 


  \item $ $
  \begin{reviewer}
    The last few paragraphs of Section I should preferably be moved to a different section on problem formulation and the problem construction must be explicitly defined. Without a problem formulation and motivation for the problem, the paper is difficult to understand. 
   \end{reviewer}

\noindent
Thanks for the advice. This was also picked up by other reviewers who also suggested a restructuring. We fully agree and have done so. We have separated the introduction 
into multiple subsections: a motivating introduction proper (Section I), related works (Section II), problem statement (Section III), and recounting of the existing state-of-the-art algorithm for the NCPP task (SectionIV). 
The motivation for solving the NCPP task is presented in further detail in Section I. 

  \item $ $
  \begin{reviewer}
Additionally, the authors should also add a list of contributions to the paper. 
  \end{reviewer}

\noindent
%Thanks for the advice. 
We have listed the contributions of this paper in the introduction as advised (Section IA).
Besides, we have additionally open-sourced a C++ implementation of the proposed IMCGS solver. 

  \item $ $
  \begin{reviewer}
   Last paragraph, page 2: The authors motivate and justify their solution strategy as a betterment of the cell-division and enumeration strategy that developed in [23]. What is that cell division? Authors use that cell division in figure 2 and it is not apparent to a reader without reading [23], what exactly is happening in figure 2. 
   \end{reviewer}

\noindent
%Thanks for the advice. 
We have provided the necessary preliminaries of earlier work and compiled them into a single Section IV in this revision for added clarity. 
In there, we have also added Fig.3 to complement the original Fig.2 (now Fig.4) to illustrate with a clear and simpler example the cell subdivision concept. 
The minimal example is analysed to present the resulting enumeration process via Eqns.(5) and (6). 

  \item $ $
  \begin{reviewer}
   It is not clear from the text if lemma 22 is a contribution of this paper. If it is, then the authors must give a proof. If not, then the authors must make it apparent in the text in the appendix. In the current form it is not clear. Also, the text of the lemma is grammatically incorrect as pointed out above. 
   \end{reviewer}

\noindent
%Thanks for the advice. 
This was also pointed out by R1.VI, please refer to the answer therein provided too.
Lemma 22 was proposed in earlier work [23], and the proof supplied there; it is not a contribution of this paper. 
We have made the citation explicit as Lemma 7  in the revised manuscript. 


\end{enumerate}

\clearpage

\section*{Response to Reviewer 3}
We thank the Reviewer for the careful examination of our response and the revised paper, and the helpful feedback and suggestions. Please see our detailed responses below.



\begin{enumerate}[I.]
 \item $ $
  \begin{reviewer}
    Introduction is too long and confusing. It is suggested to create multiple subsections in the introduction: Motivation, Challenges, Relevant Work, Proposed Method, Contributions, Paper Organization. 
    \end{reviewer}

\noindent
%Thanks for the advice. 
As also suggested by other reviewers, the paper has been reorganised to increase clarity, particularly the earlier parts as advised. 
It has been broken down into multiple sections: introduction (Section I), related works (Section II), problem statement (Section III), and recounting of the existing state-of-the-art algorithm for the NCPP task (Section IV). 

  \item $ $
  \begin{reviewer}
    The paper cites very old papers for CPP. Many new papers are missing. 
   \end{reviewer}

\noindent
%Thanks for the advice. 
The CPP problem is indeed an established research topic, and many original citations necessarily dates back some time. Having said that, it is not a heavily researched topic and references are not vast.
We have nevertheless added more recent literature citations (in the related work, Section II, and in the motivation behind the NCPP problem in Section I - this was also suggested by other reviewers). 

  \item $ $
  \begin{reviewer}
I wonder why so many conference papers are cited while their journal versions exist. 
   \end{reviewer}

Thanks. %for the advice. 
We looked further after the advice from the reviewer but could only locate three conference papers ([2] - also [2] in the original submission, [21] - [13] in the original submission, and [23] - [19 in the original submission) with a journal version counterpart, 
which we have switched in the latest revision of the manuscript. 

It is our understanding that the contributions in the field have been relatively incremental from the early days of the problem being studied, hence making it harder to reach a journal publication. It is our belief that our proposed graph representation and improved solver represents a marked novel approach when compared to other existing task-space coverage planners, as we highlight in the introduction and related works.
  
\newpage

  \item $ $
  \begin{reviewer}
  Issues on mathematical statements: 
\begin{enumerate}
\item Section 2: Please define topological invariant variable in your problem context. “A topological invariant variable ... altogether” is vague and high level. Please given precise definitions as needed. 
\item Definition 1: What class of distribution? Moreover the definition itself is confusing. 
\item Proposition 2: What graph? What cell? Homotopic cutting paths? Please define them before using them. 
\item Lemma 4: This does not sound like a mathematical statement. More like a fact. 
\item Corollary 5: This does not sound like a mathematical statement. Also, the proof is long and unclear. 
\item Lemma 6: what 0 and 1? I know they are discussed before but never defined. Also, what do you mean by “not countable”? Do you mean uncountable infinite? That is not true. 
\item Corollary 7: This does not sound like a mathematical statement. 
\item Definition 9: What is “boundary of the graph”? 
\item Theorem 11: This does not sound like a mathematical statement with a mathematical proof. 
\end{enumerate}

\end{reviewer}

\noindent
We take heed of the reviewer's comments and address them point-by-point below. 
We would also like to add that in the initial submission we aimed to intuitively introduce the difficulties in solving a topological graph with intersections in Section II, 
and then propose the actual algorithm in Section III. However, from the reviewers' feedback it looks like the intuitive analysis brought unnecessary vagueness. 
%Precisely defining and locating intersections are indeed high-level and vague. 
In contrast, in the revised manuscript we have identified more elegant ways to precisely define the intersection-free sub-graph which our work is predicated on, 
allowing us to more simply enhance the mathematical formalism and precision of the work presented. 

Following the reviewer's comments, we decided to remove some of the assertions mentioned that we agree do not directly contribute to the generic  improved solver, 
moved some of the more intuitive explanations to the captions of the illustrations supplied with the remit of gaining a better understanding of the problem, 
and avoided defining terms that are only used sporadically or restricted to given examples that can be better understood directly from the illustrations. 
We believe all these simplifications have made the paper clearer and easier to follow and comprehend. 

\noindent
%Concretely, 
\begin{enumerate}
\item We thank the reviewer for picking up on this item in particular. A topological invariant variable is indeed a vague definition in mathematical terms. 
We wanted to draw attention to the concept of topological intersections %are topological invariant variables 
in our initial submission, yet % we now realise 
% that it did not directly contribute to the proposed solver when . 
in the revised manuscript we have shifted focused not to the actual ``intersections'',  but the intersection-free property of a sub-graph, 
which can be precisely defined - see in Def. 8 to 10. 
As a result, the mathematical ambiguity that may be attributed to ``topological invariant variable" and ``intersections" is no longer relevant. 
They have been removed altogether to add clarity and a more correct mathematical representation. 
\item The colours represent disjointed subsets of valid manipulator IK configurations. 
The distribution of a colour is the reachable task-space region of the set of configurations. 
Due to the non-linear nature of manipulator kinematics, one task-space point corresponds to multiple colours, whilst in solving the NCPP task we can only choose one among them. 
%Hence the reachable region of different colours are mutually interrupted. 
We have more precisely defined preliminary terms in Section IV, but avoided using the term ``distribution of colours" to prevent the confusion expressed by the reviewer. 
\item A cell, in an analogy to its definition in the classic coverage path planning (CPP) problem, represents an area in the region to be covered. 
We had added some of these key definitions ( cell, graph, cutting path ...), formally defined in [23], as an appendix in the original submission. 
In feedback common from all reviewers, we now understand that these are better incorporated into the main body of the manuscript, and introduced earlier, and we have done so.  
They are now precisely described in the context of stating the relation between the CPP and NCPP problem at hand at the beginning of Section III, 
and in Section IV of the revised manuscript. 
Moreover, we have also added a glossary with key terms, listed as an appendix. 
As suggested, we have checked that all definitions have been defined before being used. 
%A number of definitions such as cell, graph, and cutting paths are borrowed in early work [23] so we put them in appendix. 
\item The statement of multiplicity appears trivial after the existence of intersections is assumed, so it was noteworthy at the time, if somewhat mathematically vague, we agree. 
Establishing the lemma was aimed at intuitively revealing the difficulty of collecting all optimal solutions for an NCPP task. Fig. 9 is a good illustration of the complexities involved. 
As mentioned in (a), after coming to terms with the realisation intersections can be regarded differently to address the NCPP problem, %d not directly contribute to the proposed solver. 
we have removed the lemma altogether, resulting in a more precise proposition and a paper that is easier to read. 
%The same claim can be concluded by looking at the example shown in Fig.9. 
\item Corollary 5 (together with its visual illustration in Figs.4 and 5 in the original submission, 5 and 7 in the new submission) provided an example where all optimal solutions 
had to be collected by costly, full enumeration. 
%For a generic optimal graph solver, 
One such example was sufficient to show that all optimal graph solvers must be enumerative, so it is noteworthy as part of a generic optimal graph solver. 
We note however that the explanation is specific to the example provided, and being there with the objective to illustrate the full enumeration issue, we have decided after the reviewer's comment to move the necessary explanations to the figure caption (Fig.7), and removed the non-critical definition of ``entry" to increase flow and readability. 
\item This confusion goes back to the vagueness in the definition of ``intersections" in the initial submission. 
In this revised manuscript, we have directly focused on the intersection-free property, so Lemma 6 (together with discussions to count the number of intersections in a graph, e.g., Fig.7) has been removed for clarity. 
\item In Corollary 7 we claimed the nonexistence of a kind of graph solving processes that decrease the ``number of intersections" gradually. 
We thought this is critical to note since the proposed improved solver is also an enumerative one. 
However, this is not directly related to the proposed solver, so we have removed it to make the paper easier. 
\item We have defined the boundary of a sub-graph in the new Def.8. 
\item It is indeed difficult to prove the property in Theorem 11 in a formal and precise way. 
So we have adopted a more straightforward way to do so in the revised manuscript as described above in (a), thus focusing on the intersection-free property of a sub-graph.
Please refer to  the new Def.9 in the revised manuscript. 
%As such, we directly define an intersection-free sub-graph as the sub-graphs that have the property declared in Theorem 11 (Def.9 in the revised manuscript). In that way, an intersection-free sub-graph is subject to the choice of colours of its boundary cells, and the difference between a strip and a sub-graph is that, given any possible colour of boundary cells of a strip, the strip is always an intersection-free sub-graph. 
In this way, now in Section V we present how an intersection-free sub-graph can be effectively solved, and in Section VI we propose 
the algorithm to separate a graph into intersection-free sub-graphs. 
\end{enumerate}
  
\item $ $
\begin{reviewer}
I didn’t see any section with the title “Proposed Method”. 
\end{reviewer}

\noindent
%Thanks for the advice. 
We have renamed Section V and VI as ``Proposed Method'', as the proposed solver mechanism is effectively split into two sections for increased readability.
%SVI  as ``Proposed Method: Graph Pre-Separation". 
In Section V ``Proposed Method: Optimality via Topological Intersections" we show what would happen if a sub-graph is an intersection-free sub-graph (with all its boundary cells being assigned a colour, the optimal colour assignment of its internal cells is unique), and in Section VI ``Proposed Method: Graph Pre-Separation" we show how to separate a graph into intersection-free sub-graphs. 

%concrete steps of graph separation, as

  \item $ $
  \begin{reviewer}
The experimental results don’t show any performance measures such as total time? Only number of iterations are shown.  
   \end{reviewer}

\noindent
%Thanks for the advice. 
We have added the theoretical complexity bound, actual number of enumerations, and the computational time of all case studies, and collected  them singularly in Table 1. 
Two extra case studies are provided in the revised manuscript, with animations of the resulting manipulator motions for all experimental case studies supplemented in the additional video. 
\end{enumerate}


\end{document}
