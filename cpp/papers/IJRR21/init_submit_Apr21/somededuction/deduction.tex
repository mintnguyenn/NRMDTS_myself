%%%%%%%%%%%%%%%%%%%%%%%%%%%%%%%%%%%%%%%%%%%%%%%%%%%%%%%%%%%%%%%%%%%%%%%%%%%%%%%%
%2345678901234567890123456789012345678901234567890123456789012345678901234567890
%        1         2         3         4         5         6         7         8
\documentclass[utf8]{article}
                              % Needed to meet printer requirements.

%In case you encounter the following error:
%Error 1010 The PDF file may be corrupt (unable to open PDF file) OR
%Error 1000 An error occurred while parsing a contents stream. Unable to analyze the PDF file.
%This is a known problem with pdfLaTeX conversion filter. The file cannot be opened with acrobat reader
%Please use one of the alternatives below to circumvent this error by uncommenting one or the other
%\pdfobjcompresslevel=0
%\pdfminorversion=4

% See the \addtolength command later in the file to balance the column lengths
% on the last page of the document
\usepackage{url}
\usepackage{graphics} % for pdf, bitmapped graphics files
\usepackage{epsfig} % for postscript graphics files
\usepackage{mathptmx} % assumes new font selection scheme installed
\usepackage{times} % assumes new font selection scheme installed
\usepackage{amsmath} % assumes amsmath package installed
\usepackage{amssymb}  % assumes amsmath package installed
%\usepackage{amsthm}
\usepackage{bm}
\usepackage{mathrsfs}
\usepackage{color}
\usepackage{cite}
\usepackage{threeparttable}

\usepackage{multirow}
\usepackage{bigdelim}

\usepackage{algorithm}
\usepackage{algorithmicx}
\usepackage{algpseudocode}

\usepackage{graphicx}
\usepackage{subfigure}

\usepackage{geometry}
\geometry{a4paper, scale=0.9}

%\usepackage[all]{xy}

\newtheorem{theorem}{Theorem}[section]
\newtheorem{definition}{Definition}[section]
\newtheorem{lemma}{Lemma}[section]
\newtheorem{remark}{Remark}[section]
\newtheorem{corollary}{Corollary}[section]
\newtheorem{notation}{Notation}[section]
\newtheorem{problem}{Problem}[section]

\floatname{algorithm}{Algorithm}
\renewcommand{\algorithmicrequire}{\textbf{Input:}}
\renewcommand{\algorithmicensure}{\textbf{Output:}}


\author{Tong Yang, Yue Wang and Rong Xiong% <-this % stops a space
%\thanks{*This work was not supported by any organization}% <-this % stops a space
\thanks{Tong Yang, Yue Wang and Rong Xiong are with the State Key Laboratory of Industrial Control and Technology, Zhejiang University, P.R. China. Yue Wang is the corresponding author {\tt\small wangyue@iipc.zju.edu.cn}. Rong Xiong is the co-corresponding author {\tt\small rxiong@zju.edu.cn}.}%
%\thanks{$^{2}$Bernard D. Researcheris with the Department of Electrical Engineering, Wright State University,
%        Dayton, OH 45435, USA
%        {\tt\small b.d.researcher@ieee.org}}%
}


\begin{document}

\thispagestyle{empty}
\pagestyle{empty}


%%%%%%%%%%%%%%%%%%%%%%%%%%%%%%%%%%%%%%%%%%%%%%%%%%%%%%%%%%%%%%%%%%%%%%%%%%%%%%%%

\section{Problem and Parameter}
\noindent
\indent
Consider a 5DOF Manipulator holding the end-effector (EE) on the given 3D position and pointing (EE's z-axis) at the given direction. D-H parameter is given as follow
\begin{table}[h]
\centering
\begin{tabular}{|c|c|c|c|c|}
\hline
$i$ & $\alpha_i$ & $a_i$ & $\theta_i$ & $d_i$ \\
\hline
\hline
1 & $\pi/2$ & 0 & $\theta_1$ & 0.089159 \\
\hline
2 & 0 & $-0.425$ & $\theta_2$ & 0 \\
\hline
3 & 0 & $-0.39225$ & $\theta_3$ & 0 \\
\hline
4 & $\pi/2$ & $0$ & $\theta_4$ & 0.10915 \\
\hline
5 & $-\pi/2$ & 0 & $\theta_5$ & 0.09465 \\
\hline 
6 & 0 & 0 & $\theta_6$ & 0.0823 \\
\hline
\end{tabular}

\end{table}

Target pose is 
$$(p_x, p_y, p_z, \vec{n}),\ \vec{n} = [n_x, n_y, n_z]^T$$
where $\vec{n}$ is a normal vector of the surface of the object. 

Since the end-effector should be perpendicular to the surface, 
\begin{equation}
^0T_{ee} = \left[
\begin{matrix}
a_x & b_x & n_x & p_x\\
a_y & b_y & n_y & p_y\\
a_z & b_z & n_z & p_z\\
0 & 0 & 0 & 1
\end{matrix}
\right]
\end{equation}

where $a_x, a_y, a_z, b_x, b_y, b_z$ are all variables, and
\begin{equation}\label{constraints}
\left\{
\begin{aligned}
a_xn_x + a_yn_y + a_zn_z &= 0\\
b_xn_x + b_yn_y + b_zn_z &= 0\\
a_xb_x + a_yb_y + a_zb_z &= 0\\
\parallel a\parallel_2 &= 1\\
\parallel b\parallel_2 &= 1
\end{aligned}
\right.
\end{equation}


\section{Kinematics}
The kinematic function of the manipulator: 
$$T = ^0T_1 ^1T_2\cdots ^4T_5^5T_{ee}$$
where
\begin{equation}
^0T_1 = \left[
\begin{matrix}
c\theta_1 & 0 & s\theta_1  & 0\\
s\theta_1 & 0 & -c\theta_1 & 0\\
0         & 1 & 0          & d_1\\
0         & 0 & 0          & 1
\end{matrix}
\right],\ 
^1T_2 = \left[
\begin{matrix}
c\theta_2 & -s\theta_2 & 0 & a_2c\theta_2\\
s\theta_2 & c\theta_2  & 0 & a_2s\theta_2\\
0         & 0          & 1 & 0\\
0         & 0          & 0 & 1
\end{matrix}
\right],\ 
^2T_3 = \left[
\begin{matrix}
c\theta_3 & -s\theta_3 & 0 & a_3c\theta_3\\
s\theta_3 & c\theta_3 & 0 & a_3s\theta_3\\
0 & 0 & 1 & 0\\
0 & 0 & 0 & 1
\end{matrix}
\right]
\end{equation}
\begin{equation}
^3T_4 = \left[
\begin{matrix}
c\theta_4 & 0 &s\theta_4 & 0\\
s\theta_4 & 0 & -c\theta_4 & 0\\
0 & 1 & 0 & d_4\\
0 & 0 & 0 & 1
\end{matrix}
\right],\ 
^4T_5 = \left[
\begin{matrix}
c\theta_5 & 0 & -s\theta_5 & 0\\
s\theta_5 & 0 & c\theta_5 & 0\\
0 & -1 & 0 & d_5\\
0 & 0 & 0 & 1
\end{matrix}
\right],\ 
^5T_{ee} = \left[
\begin{matrix}
1 & 0 & 0 & 0\\
0 & 1 & 0 & 0\\
0 & 0 & 1 & d_{ee}\\
0 & 0 & 0 & 1
\end{matrix}
\right]
\end{equation}

To simplify the calculation, we get 
\begin{equation}
\begin{aligned}
^1T_3 = ^1T_2^2T_3 &= \left[
\begin{matrix}
c\theta_2c\theta_3 - s\theta_2s\theta_3 & -c\theta_2s\theta_3-c\theta_3s\theta_2 & 0 & a_2c\theta_2 + a_3(c\theta_2c\theta_3 - s\theta_2s\theta_3)\\
c\theta_2s\theta_3 + c\theta_3s\theta_2 & c\theta_2c\theta_3 - s\theta_2s\theta_3 & 0 & a_2s\theta_2 + a_3(c\theta_2s\theta_3 + a_3c\theta_3s\theta_2)\\
0 & 0 & 1 & 0\\
0 & 0 & 0 & 1
\end{matrix}
\right]\\
&\triangleq \left[
\begin{matrix}
c_{23} & -s_{23} & 0 & a_3c_{23}+ a_2 c_2\\
s_{23} & c_{23} & 0 & a_3 s_{23} + a_2s_2 \\
0 & 0 & 1 & 0\\
0 & 0 & 0 & 1
\end{matrix}
\right]
\end{aligned}
\end{equation}

\begin{equation}
^1T_4 = ^1T_3^3T_4 = \left[
\begin{matrix}
c_{23}c\theta_4 - s_{23}s\theta_4 & 0 & c_{23}s\theta_4 + s_{23}c\theta_4 & a_3c_{23} + a_2c\theta_2\\
s_{23}c\theta_4 + c_{23}s\theta_4 & 0 & s_{23}s\theta_4 - c_{23}c\theta_4 & a_3s_{23} + a_2s\theta_2\\
0 & 1 & 0 & d_4\\
0 & 0 & 0 & 1
\end{matrix}
\right]
\end{equation}

\begin{equation}\label{1T5_1}
^1T_5 = ^1T_4^4T_5 = \left[
\begin{matrix}
(c_{23}c\theta_4 - s_{23}s\theta_4)c\theta_5 & -c_{23}s\theta_4 - s_{23}c\theta_4 & (-c_{23}c\theta_4 + s_{23}s\theta_4)s\theta_5 & d_5(c_{23}s\theta_4 + s_{23}c\theta_4) + a_3c_{23}+ a_2c_2\\
(s_{23}c\theta_4 + c_{23}s\theta_4)c\theta_5 & -s_{23}s\theta_4 + c_{23}c\theta_4 & (-s_{23}c\theta_4 - c_{23}s\theta_4)s\theta_5 & d_5(s_{23}s\theta_4 - c_{23}c\theta_4) + a_3s_{23}+ a_2s\theta_2\\
s\theta_5 & 0 & c\theta_5 & d_4\\
0 & 0 & 0 & 1 
\end{matrix}
\right]
\end{equation}




We can precalculate the $^0T_1$ and $^5T_{ee}$ to simplify the calculation
\begin{equation}
(^0T_1)^{-1} = \left[
\begin{matrix}
c\theta_1 & s\theta_1 & 0 & 0\\
0 & 0 & 1 & -d_1\\
s\theta_1 & -c\theta_1 & 0 & 0\\
0 & 0 & 0 & 1
\end{matrix}
\right],\ 
(^5T_{ee})^{-1} = \left[
\begin{matrix}
1 & 0 & 0 & 0\\
0 & 1 & 0 & 0\\
0 & 0 & 1 & -d_{ee}\\
0 & 0 & 0 & 1
\end{matrix}
\right]
\end{equation}
So we get
\begin{equation}\label{1T5_2}
\begin{aligned}
^1T_5 &= ((^0T_1)^{-1}) ^0T_{ee}((^5T_{ee})^{-1})\\
&= \left[
\begin{matrix}
c\theta_1 & s\theta_1 & 0 & 0\\
0 & 0 & 1 & -d_1\\
s\theta_1 & -c\theta_1 & 0 & 0\\
0 & 0 & 0 & 1
\end{matrix}
\right]\left[
\begin{matrix}
a_x & b_x & n_x & p_x\\
a_y & b_y & n_y & p_y\\
a_z & b_z & n_z & p_z\\
0 & 0 & 0 & 1
\end{matrix}
\right]\left[
\begin{matrix}
1 & 0 & 0 & 0\\
0 & 1 & 0 & 0\\
0 & 0 & 1 & -d_{ee}\\
0 & 0 & 0 & 1
\end{matrix}
\right]\\
&= \left[
\begin{matrix}
c\theta_1a_x+ s\theta_1a_y & c\theta_1b_x + s\theta_1b_y & c\theta_1n_x+ s\theta_1n_y & c\theta_1p_x+ s\theta_1 p_y\\
a_z & b_z & n_z & p_z - d_1\\
s\theta_1 a_x- c\theta_1a_y & s\theta_1b_x - c\theta_1b_y & s\theta_1n_x - c\theta_1 n_y & s\theta_1 p_x - c\theta_1 p_y\\
0 & 0 & 0 & 1
\end{matrix}
\right]\left[
\begin{matrix}
1 & 0 & 0 & 0\\
0 & 1 & 0 & 0\\
0 & 0 & 1 & -d_{ee}\\
0 & 0 & 0 & 1
\end{matrix}
\right]\\
&= \left[
\begin{matrix}
c\theta_1 a_x+ s\theta_1a_y & c\theta_1b_x+ s\theta_1b_y & c\theta_1n_x + s\theta_1n_y & -d_{ee}(c\theta_1n_x + s\theta_1n_y) + c\theta_1p_x + s\theta_1 p_y\\
a_z & b_z & n_z & -d_{ee}n_z + p_z - d_1\\
s\theta_1a_x - c\theta_1a_y & s\theta_1b_x - c\theta_1b_y & s\theta_1n_x-c\theta_1n_y & -d_{ee}(s\theta_1n_x - c\theta_1n_y) + s\theta_1p_x - c\theta_1p_y\\
0 & 0 & 0 & 1
\end{matrix}
\right]
\end{aligned}
\end{equation}



\subsection{Solve $\theta_1$}
From $(3, 4)$ in Equ. \ref{1T5_1} and Equ.\ref{1T5_2}, we know
$$d_4 = -d_{ee}(s\theta_1n_x - c\theta_1n_y) + s\theta_1p_x - c\theta_1p_y$$
$$\frac{p_x - d_{ee}n_x}{\sqrt{(p_x - d_{ee}n_x)^2 + (d_{ee}n_y - p_y)^2}}s\theta_1 + \frac{d_{ee}n_y - p_y}{\sqrt{(p_x - d_{ee}n_x)^2 + (d_{ee}n_y - p_y)^2}}c\theta_1 = \frac{d_4}{\sqrt{(p_x - d_{ee}n_x)^2 + (d_{ee}n_y - p_y)^2}}$$
Set $\phi$ that 
$$\tan\phi = \frac{p_x - d_{ee}n_x}{d_{ee}n_y - p_y} \Rightarrow \phi = {\rm atan2}\left(\frac{p_x - d_{ee}n_x}{d_{ee}n_y - p_y}\right)$$
$$\cos(\theta_1 - \phi) = \sin\phi\sin\theta_1 + \cos\phi\cos\theta_1 = \frac{d_4}{\sqrt{(p_x - d_{ee}n_x)^2 + (d_{ee}n_y - p_y)^2}}$$
$$\Rightarrow \theta_1 - \phi = {\rm acosall}\left(\frac{d_4}{\sqrt{(p_x - d_{ee}n_x)^2 + (d_{ee}n_y - p_y)^2}}\right)$$
\begin{equation}
\theta_1 = {\rm acosall}\left(\frac{d_4}{\sqrt{(p_x - d_{ee}n_x)^2 + (d_{ee}n_y - p_y)^2}}\right) + {\rm atan2}\left(p_x - d_{ee}n_x, d_{ee}n_y - p_y\right)
\end{equation}

\subsection{Solve $\theta_5$}
From $(3, 3)$ in Equ. \ref{1T5_1} and Equ.\ref{1T5_2}, we know
$$c\theta_5 = s\theta_1n_x - c\theta_1 n_y$$
\begin{equation}
\theta_5 = {\rm acosall}(s\theta_1n_x - c\theta_1 n_y)
\end{equation}

\subsection{Solve $b_x, b_y, b_z$}
\begin{color}{blue}
Compared to the derivation in Tmech, we do following improvements: 
\subsubsection{If $n_z \neq 0$}
From $(3, 2)$ in Equ. \ref{1T5_1} and Equ.\ref{1T5_2} and Equ. \ref{constraints}, we know
$$\left\{
\begin{aligned}
s\theta_1b_x - c\theta_1b_y &= 0\\
b_xn_x + b_yn_y + b_zn_z &= 0\\
b_x^2 + b_y^2 + b_z^2 = 1
\end{aligned}
\right.
\Rightarrow 
\left\{
\begin{aligned}
&b_y = \frac{s\theta_1}{c\theta_1}b_x\\
&\frac{n_x}{n_z}b_x + \frac{n_y}{n_z}b_y = -b_z
\end{aligned}
\right.$$
$$b_x^2 + \left(\frac{s\theta_1}{c\theta_1}b_x\right)^2 + \left(\frac{n_x}{n_z}b_x + \frac{n_y}{n_z}\frac{s\theta_1}{c\theta_1}b_x\right)^2 = 1$$
$$b_x^2\left( 1 + \frac{s\theta_1^2}{c\theta_1^2} + \frac{n_x^2}{n_z^2} + \frac{n_y^2s\theta_1^2}{n_z^2c\theta_1^2} + 2\frac{n_xn_ys\theta_1}{n_z^2c\theta_1} \right) = 1$$
\begin{equation}
\left\{
\begin{aligned}
&b_x =\pm \sqrt{\frac{1}{1 + \frac{s\theta_1^2}{c\theta_1^2} + \frac{n_x^2}{n_z^2} + \frac{n_y^2s\theta_1^2}{n_z^2c\theta_1^2} + 2\frac{n_xn_ys\theta_1}{n_z^2c\theta_1}}}\\
&b_y = \frac{s\theta_1}{c\theta_1}b_x\\
&bz = -\frac{n_x}{n_z}b_x - \frac{n_y}{n_z}b_y
\end{aligned}
\right.
\end{equation}
\subsubsection{If $n_z = 0$}
From $(3, 2)$ in Equ. \ref{1T5_1} and Equ.\ref{1T5_2} and Equ. \ref{constraints}, we know
$$\left\{
\begin{aligned}
s\theta_1b_x - c\theta_1b_y &= 0\\
b_xn_x + b_yn_y + b_zn_z &= 0\\
b_x^2 + b_y^2 + b_z^2 &= 1
\end{aligned}
\right.
\Rightarrow 
\left\{
\begin{aligned}
s\theta_1b_x - c\theta_1b_y &= 0\\
b_xn_x + b_yn_y &= 0\\
b_x^2 + b_y^2 + b_z^2 &= 1
\end{aligned}
\right.
\Rightarrow 
\left\{
\begin{aligned}
&b_y = \frac{s\theta_1}{c\theta_1}b_x\\
&(s\theta_1 = n_x, c\theta_1 = -n_y \mbox{ or opposite})
\end{aligned}
\right.$$
Hence
$$\left\{
\begin{aligned}
&b_x = \pm c\theta_1\sqrt{1-b_z^2}\\
&b_y = \frac{s\theta_1}{c\theta_1}b_x\\
&b_z = \forall \in (-1, 1)
\end{aligned}
\right.
\mbox{ or }
\left\{
\begin{aligned}
&b_x = 0\\
&b_y = 0\\
&b_z = -1 \mbox{ or } 1
\end{aligned}
\right.
$$

\end{color}

\subsection{Solve $a_x, a_y, a_z$}
Since $\vec{a} = \vec{b}\times \vec{n}$, 
\begin{equation}
\left\{
\begin{aligned}
a_x &= b_yn_z - b_zn_y\\
a_y &= b_zn_x - b_xn_z\\
a_z &= b_xn_y - b_yn_x
\end{aligned}
\right.
\end{equation}
From $(3, 1)$ in the Equ. \ref{1T5_2} we must check
$$s\theta_5 = s\theta_1a_x - c\theta_1 a_y$$

\subsection{Balance Equation}
$$ ^1T_2^2T_3^3T_4 = (^0T_1)^{-1}(^0T_{ee})(^5T_{ee})^{-1}(^4T_5)^{-1}$$
\begin{equation}
^4T_5^{-1} = \left[
\begin{matrix}
c_5 & s_5 & 0 & 0\\
0 & 0 & -1 & d_5\\
-s_5 & c_5 & 0 & 0\\
0 & 0 & 0 & 1
\end{matrix}
\right]
\end{equation}
\begin{equation}\label{1T4}
Left = \left[
\begin{matrix}
c_{234} & 0 & s_{234} & a_3c_{23}+a_2c_2\\
s_{234} & 0 & -c_{234} & a_3s_{23}+a_2s_2\\
0 & 1 & 0 & d_4\\
0 & 0 & 0 & 1
\end{matrix}
\right],\ Right = \left[
\begin{matrix}
Right_{11} & Right_{12} \\
Right_{21} & Right_{22} 
\end{matrix}
\right]
\end{equation}
where
$$Right_{11} = \left[
\begin{matrix}
c\theta_5(c\theta_1a_x +s\theta_1a_y) - s\theta_5(c\theta_1n_x + s\theta_1n_y) & s\theta_5(c\theta_1a_x +s\theta_1a_y) + c\theta_5(c\theta_1n_x + s\theta_1n_y) \\
c\theta_5a_z - s\theta_5n_z & s\theta_5a_z + c\theta_5n_z
\end{matrix}
\right]$$
$$Right_{12} = \left[
\begin{matrix}
-c\theta_1b_x - s\theta_1b_y & d_5(c\theta_1b_x + s\theta_1b_y) - d_{ee}(c\theta_1n_x + s\theta_1n_y) + c\theta_1p_x + s\theta_1p_y\\
-b_z & d_5b_z - d_{ee}n_z + p_z - d_1
\end{matrix}
\right]$$
$$Right_{21} = \left[
\begin{matrix}
c\theta_5(s\theta_1a_x - c\theta_1a_y) - s\theta_5(s\theta_1n_x - c\theta_1n_y) & s\theta_5(s\theta_1a_x - c\theta_1a_y) + c\theta_5(s\theta_1n_x - c\theta_1n_y)\\
0 & 0
\end{matrix}
\right]$$
$$Right_{22} = \left[
\begin{matrix}
-s\theta_1b_x + c\theta_1b_y & d_5(s\theta_1b_x - c\theta_1b_y) - d_{ee}(s\theta_1n_x - c\theta_1n_y) + s\theta_1p_x - c\theta_1p_y\\
0 & 1
\end{matrix}
\right]$$

\subsection{Solve $\theta_3$ }
From $(1, 4)$ and $(2, 4)$ in Equ. \ref{1T4} we know (and define $m_1$, $m_2$ as)
\begin{equation}\label{check23}
\left\{
\begin{aligned}
m_1 &\triangleq d_5(c\theta_1b_x + s\theta_1b_y) - d_{ee}(c\theta_1n_x + s\theta_1n_y) + c\theta_1p_x + s\theta_1p_y = a_3c_{23}+ a_2c_2\\
m_2 &\triangleq d_5b_z - d_{ee}n_z + p_z - d_1 = a_3s_{23}+ a_2s_2
\end{aligned}
\right.
\end{equation}
$$\left\{
\begin{aligned}
&m_1^2 = a_3^2c_{23}^2 + a_2^2c_2^2 + 2a_2a_3c_2c_{23}\\
&m_2^2 = a_3^2s_{23}^2 + a_2^2s_2^2 + 2a_2a_3s_2s_{23}
\end{aligned}
\right.\Rightarrow m_1^2 + m_2^2 = a_3^2 + a_2^2 + 2a_2a_3(c_2c_{23} + s_2s_{23})
$$
\begin{equation}
\theta_3 = {\rm acosall}(\frac{m_1^2 + m_2^2 - a_3^2 - a_2^2}{2a_2a_3})
\end{equation}


\subsection{Solve $\theta_2$}
%$$\left\{
%\begin{aligned}
%&a_3(c\theta_2c\theta_3 - s\theta_2s\theta_3) + a_2c\theta_2 = m_1\\
%&a_3(s\theta_2c\theta_3 + c\theta_2s\theta_3) + a_2s\theta_2 = m_2
%\end{aligned}
%\right.\Rightarrow \left\{
%\begin{aligned}
%&a_2 + a_3c\theta_3 - a_3s\theta_3t\theta_2 = \frac{m_1}{c\theta_2}\\
%&a_3s\theta_3+ (a_2 + a_3c\theta_3)t\theta_2 = \frac{m_2}{c\theta_2}
%\end{aligned}
%\right.\Rightarrow \frac{m_1}{m_2} = \frac{a_2 + a_3c\theta_3 - a_3s\theta_3t\theta_2}{a_3s\theta_3+ (a_2 + a_3c\theta_3)t\theta_2}$$
%$$m_1a_3s\theta_3 + m_1(a_2+a_3c\theta_3)t\theta_2 = m_2(a_2 + a_3c\theta_3) - m_2a_3s\theta_3t\theta_2$$
%$$(m_1(a_2+a_3c\theta_3) + m_2a_3s\theta_3)t\theta_2 = m_2(a_2 + a_3c\theta_3) - m_1a_3s\theta_3$$
%\begin{equation}
%\theta_2 = {\rm atanall}( \frac{m_2(a_2 + a_3c\theta_3) - m_1a_3s\theta_3}{m_1(a_2+a_3c\theta_3) + m_2a_3s\theta_3} )
%\end{equation}
%
%\subsubsection{Another solution}
From Equ. \ref{check23} we know
$$\left\{
\begin{aligned}
&a_3(c\theta_2c\theta_3 - s\theta_2s\theta_3) + a_2c\theta_2 = m_1\\
&a_3(s\theta_2c\theta_3 + c\theta_2s\theta_3) + a_2s\theta_2 = m_2
\end{aligned}
\right.\Rightarrow 
\left\{
\begin{aligned}
&\frac{a_3c\theta_3+a_2}{a_3s\theta_3}c\theta_2 - s\theta_2 = \frac{m_1}{a_3s\theta_3}\\
&\frac{a_3s\theta_3}{a_3c\theta_3 + a_2}c\theta_2 + s\theta_2 = \frac{m_2}{a_3c\theta_3 +a_2}
\end{aligned}
\right.
$$
$$(\frac{a_3c\theta_3+a_2}{a_3s\theta_3} + \frac{a_3s\theta_3}{a_3c\theta_3 + a_2})c\theta_2 = \frac{m_1}{a_3s\theta_3} +\frac{m_2}{a_3c\theta_3 +a_2}$$
\begin{equation}
\theta_2 = {\rm acosall}\frac{\frac{m_1}{a_3s\theta_3} +\frac{m_2}{a_3c\theta_3 +a_2}}{\frac{a_3c\theta_3+a_2}{a_3s\theta_3} + \frac{a_3s\theta_3}{a_3c\theta_3 + a_2}}
\end{equation}

After solving $\theta_2$ and $\theta_3$, we should check whether $\theta_2$ and $\theta_3$ satisfy the equation \ref{check23}, because we have used the square function which will lead to multiple solutions.


\subsection{Solve $\theta_4$}
From $(1, 1)$ and $(2, 1)$ in Equ. \ref{1T4} we know
$$\left\{
\begin{aligned}
&s_{234} = c\theta_5a_z - s\theta_5n_z\\
&c_{234} = c\theta_5(c\theta_1a_x +s\theta_1a_y) - s\theta_5(c\theta_1n_x + s\theta_1n_y)
\end{aligned}
\right.$$
\begin{equation}
\theta_4 = -\theta_2 - \theta_3 + {\rm atan2}(c\theta_5a_z - s\theta_5n_z, c\theta_5(c\theta_1a_x +s\theta_1a_y) - s\theta_5(c\theta_1n_x + s\theta_1n_y)
\end{equation}

\section{Summary}
\begin{equation*}
\theta_1 = {\rm acosall}\left(\frac{d_4}{\sqrt{(p_x - d_{ee}n_x)^2 + (d_{ee}n_y - p_y)^2}}\right) + {\rm atan2}\left(p_x - d_{ee}n_x, d_{ee}n_y - p_y\right)
\end{equation*}
\begin{equation*}
\theta_5 = {\rm acosall}(s\theta_1n_x - c\theta_1 n_y)
\end{equation*}
\begin{equation*}
b_x =\pm \sqrt{\frac{1}{1 + \frac{s\theta_1^2}{c\theta_1^2} + \frac{n_x^2}{n_z^2} + \frac{n_y^2s\theta_1^2}{n_z^2c\theta_1^2} + 2\frac{n_xn_ys\theta_1}{n_z^2c\theta_1}}}
\end{equation*}
\begin{equation*}
b_y = \frac{s\theta_1}{c\theta_1}b_x
\end{equation*}
\begin{equation*}
bz = -\frac{n_x}{n_z}b_x - \frac{n_y}{n_z}b_y
\end{equation*}
\begin{equation*}
\left\{
\begin{aligned}
a_x &= b_yn_z - b_zn_y\\
a_y &= b_zn_x - b_xn_z\\
a_z &= b_xn_y - b_yn_x
\end{aligned}
\right.
\end{equation*}
%\begin{equation*}
%a_x = \pm \sqrt{\frac{1}{1 + \frac{(n_xb_z - n_zb_x)^2}{(n_yb_z - n_zb_y)^2} + \left( \frac{b_x}{b_z} - \frac{b_y(n_xb_z - n_zb_x)}{b_z(n_yb_z - n_zb_y)} \right)^2}}
%\end{equation*}
%\begin{equation*}
%a_y = -\frac{n_xb_z - n_zb_x}{n_yb_z - n_zb_y}a_x
%\end{equation*}
%\begin{equation*}
%a_z = -\frac{b_x}{b_z}a_x - \frac{b_y}{b_z}a_y
%\end{equation*}
\begin{equation*}
m_1 \triangleq d_5(c\theta_1b_x + s\theta_1b_y) - d_{ee}(c\theta_1n_x + s\theta_1n_y) + c\theta_1p_x + s\theta_1p_y
\end{equation*}
\begin{equation*}
m_2 \triangleq d_5b_z - d_{ee}n_z + p_z - d_1
\end{equation*}

\begin{equation*}
\theta_3 = {\rm acosall}(\frac{m_1^2 + m_2^2 - a_3^2 - a_2^2}{2a_2a_3})
\end{equation*}
\begin{equation*}
\theta_2 = {\rm acosall}\frac{\frac{m_1}{a_3s\theta_3} +\frac{m_2}{a_3c\theta_3 +a_2}}{\frac{a_3c\theta_3+a_2}{a_3s\theta_3} + \frac{a_3s\theta_3}{a_3c\theta_3 + a_2}}
\end{equation*}
\begin{equation*}
\theta_4 = -\theta_2 - \theta_3 + {\rm atan2}(c\theta_5a_z - s\theta_5n_z, c\theta_5(c\theta_1a_x +s\theta_1a_y) - s\theta_5(c\theta_1n_x + s\theta_1n_y)
\end{equation*}

\section{Exceptional case when $(p_x - d_{ee}n_x)^2 + (d_{ee}n_y - p_y)^2 = 0$}
\subsection{Solve $\theta_1$}
From $(3, 4)$ in Equ. \ref{1T5_1} and Equ.\ref{1T5_2}, we know
$$d_4 = -d_{ee}(s\theta_1n_x - c\theta_1n_y) + s\theta_1p_x - c\theta_1p_y$$
But we cannot normalize the equation because $(p_x - d_{ee}n_x)^2 + (d_{ee}n_y - p_y)^2 = 0$. 

\section{A singular configuration case}
Consider following EE pose 
\begin{equation}
(p_x, p_y, p_z, n_x, n_y, n_z) = [0.5285, 0.1091, 0.1757, 0, 1, 0], d_{ee} = 0.09
\end{equation}
And we go through each equation to check the validity. 



%%%%%%%%%%%%%%%%%%%%%%%%%%%%%%%%%%%%%%%%%%%%%%%%%%%%%%%%%%%%%%%%%%%%%%%%%%%%%%%%

\bibliographystyle{ieeetr} %% setting the cite style
\bibliography{paper_new}

\end{document}